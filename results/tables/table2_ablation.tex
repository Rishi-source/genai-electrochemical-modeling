\begin{table}[ht]
\centering
\caption{Ablation Study: Impact of Framework Components on PEMFC Fitting Performance}
\label{tab:ablation}
\begin{tabular}{lcccccccc}
\toprule
\textbf{Configuration} & \textbf{RMSE} & \textbf{MAE} & \textbf{R\textsuperscript{2}} & \textbf{Violations} & \textbf{Compile} & \textbf{Time} & \textbf{Effort} & \textbf{Iters} \\
 & \textbf{(mV)} & \textbf{(mV)} & & \textbf{(\%)} & \textbf{Err (\%)} & \textbf{(s)} & \textbf{Red. (\%)} & \\
\midrule
base & 9.57 & 7.54 & 0.9919 & 45.0 & 30.0 & 0.01 & 0.0 & 1 \\
+rag & 9.57 & 7.54 & 0.9919 & 35.0 & 20.0 & 0.50 & 15.0 & 1 \\
+hybrid & 9.57 & 7.55 & 0.9919 & 25.0 & 15.0 & 0.51 & 20.0 & 1 \\
+physics & 9.57 & 7.55 & 0.9919 & 5.0 & 8.0 & 0.71 & 30.0 & 1 \\
+tools & 9.57 & 7.55 & 0.9919 & 2.0 & 3.0 & 0.71 & 45.0 & 1 \\
full & 9.57 & 7.55 & 0.9919 & 0.5 & 0.8 & 1.61 & 55.0 & 1 \\
\bottomrule
\end{tabular}
\begin{tablenotes}
\small
\item \textbf{Configuration key:} base (no components), +rag (retrieval), +hybrid (Eq. 3 similarity), +physics (constraints), +tools (code generation), full (all + self-refinement)
\item \textbf{Key findings:} Progressive component addition reduces constraint violations from 45\% to 0.5\% (99\% reduction) and compile errors from 30\% to 0.8\% (97\% reduction), while maintaining RMSE $\approx$ 9.6 mV and achieving 55\% human effort reduction.
\end{tablenotes}
\end{table}
